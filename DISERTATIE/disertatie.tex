\documentclass[a4paper, 12pt]{article}
\usepackage{outlines}
\usepackage{graphicx}
%\usepackage{times}
\usepackage{xcolor}
\usepackage{float}
\usepackage{mathtools}
\usepackage{enumerate}
\usepackage{geometry}
\usepackage[english]{babel}
\usepackage{array}
\usepackage{wrapfig}
\usepackage{amsmath}
\usepackage{multirow}
\usepackage{tabulary}
\usepackage{titlesec}
\usepackage{titletoc}
\usepackage{array}
\usepackage{tabularx,ragged2e,booktabs,caption}
\usepackage{chngcntr}
\usepackage{pgf-pie} 
\usepackage{colortbl} 


\counterwithin{figure}{section}
\counterwithin{table}{section}

\titleformat{\section}[display]%
{\null\fontsize{16}{5}\rmfamily\bfseries\filcenter}{CAPITOLUL \thesection}{1em}{}[]
\titleformat{name = \section, numberless}[block]%
{\null\fontsize{16}{5}\rmfamily\bfseries\filcenter}{}{1em}{}
\titlespacing*{\section}{0em}{1em}{1em}

\titlecontents{section}
[7em] % ie, width of contentslabel + 0.5em
{\medskip}
{\contentslabel[\MakeUppercase\chaptername~\thecontentslabel]{7.5em}}%\thecontentslabel
{\hspace*{-6.5em}}
{\titlerule*[0.5pc]{.}\contentspage}

\addto{\captionsenglish}{
	\renewcommand{\refname}{Referințe bibliografice}%
	\renewcommand{\contentsname}{Cuprins}
	\renewcommand{\listfigurename}{Lista imaginilor}
	\renewcommand{\listtablename}{Lista tabelelor}
}
\addto\captionsenglish{\renewcommand{\chaptername}{Capitolul}}
%define the page geometry
\geometry{
	a4paper,
	left=25mm,
	top=25mm,
	bottom=25mm,
	right=25mm,
}
\renewcommand{\baselinestretch}{1.5}
\makeatletter
\renewcommand\paragraph{\@startsection{paragraph}{4}{\z@}%
	{-2.5ex\@plus -1ex \@minus -.25ex}%
	{1.25ex \@plus .25ex}%
	{\normalfont\normalsize\bfseries}}
\makeatother
\setcounter{secnumdepth}{4} % how many sectioning levels to assign numbers to
\setcounter{tocdepth}{4}    % how many sectioning levels to show in ToC

\begin{document}

%define the title page
\begin{titlepage}
	\begin{center}
		\vspace{0.5cm}
		\large {UNIVERSITATEA "BABEȘ-BOLYAI" CLUJ-NAPOCA}
		
		\large {FACULTATEA DE STUDII EUROPENE}
		\\
		
		
		\vspace{6cm}
		
		\Huge \textbf{LUCRARE DE DISERTAȚIE}
		
		\vspace{2 cm}
		
		\vfill
	\end{center}
	
	\begin{flushleft}
		\large{\textit{Coordonator științific:}} \\
		\large{Conf. univ. dr. Nicoleta Dorina Racolța-Paina}
	\end{flushleft}
	
	\begin{flushright}
		\hfill \large {\textit{Absolvent:}} \\
		\hfill \large {Sorina-Elena Ciucanu}
	\end{flushright}
	
	\begin{center}
		\vspace{1.5cm}
		\large \textbf{2023}
	\end{center}
\end{titlepage}
\begin{titlepage}
	\begin{center}
		\vspace{0.5cm}
		\large {UNIVERSITATEA "BABEȘ-BOLYAI" CLUJ-NAPOCA}
		\\
		\large {FACULTATEA DE STUDII EUROPENE}
		\\\large {SPECIALIZAREA MANAGEMENT PERFORMANT}
		
		\vspace{2.75cm}
		
	
		\huge Abilitățile(skills) leaderilor de echipă - de la teorie la practica. 
		
		\huge O cercetare cantitativa  in domeniul IT
		\vspace{1.5 cm}
		
		\vfill
	\end{center}
	
	\begin{flushleft}
		\large{\textit{Coordonator științific:}} \\
		\large{Conf. univ. dr. Nicoleta Dorina Racolța-Paina}
	\end{flushleft}
	
	\begin{flushright}
		\hfill \large {\textit{Absolvent:}} \\
		\hfill \large {Sorina-Elena Ciucanu}
	\end{flushright}
	
	\begin{center}
		\vspace{1.5cm}
		\Large{Cluj Napoca}
		
		\large \textbf{2021}
	\end{center}
\end{titlepage}
\restoregeometry
\thispagestyle{empty}
\section*{Declarație}
\bigskip
\qquad Prin prezenta declar că Lucrarea de licenţă cu titlul "Studiu de caz" este scrisă de mine şi nu a mai fost prezentată niciodată la o altă facultate sau instituţie de învăţământ superior din ţară sau străinătate. De asemenea, declar că toate sursele utilizate, inclusive cele de pe Internet, sunt indicate în lucrare, cu respectarea regulilor de evitare a plagiatului:
\begin{enumerate}[-]
	\item toate fragmentele de text reproduse exact, chiar şi în traducere proprie din altă limbă, sunt scrise între ghilimele şi deţin referinţa precisă a sursei;
	\item reformularea în cuvinte proprii a textelor scrise de către alţi autori deţine referinţa precisă;
	\item rezumarea ideilor altor autori deţine referinţa precisă la textul original.
\end{enumerate}

\vspace{3 cm}
\begin{flushleft}
	\large Cluj Napoca,
\end{flushleft}


\begin{flushright}
	\hfill \large Absolvent \textit{Sorina-Elena Ciucanu} \\
	\hfill \large {(semnătura olograf)}
\end{flushright}
\newpage
\thispagestyle{empty}
%put the table of contents
\tableofcontents
\thispagestyle{empty}
%put the list of figures
\newpage
\thispagestyle{empty}
\listoffigures
\newpage
\thispagestyle{empty}
\listoftables


%this section is on new page
\newpage
%\nocite{morgeson2010leadership}



	\section*{Introducere}
\addcontentsline{toc}{section}{\textsc{Introducere}}

\newpage
	\setcounter{section}{0}
	\section{Team leader-ul in organizatiile contemporane, Oportunitati si provocari}
\quad \quad\space Într-un context dinamic și competitiv, în care eficiența colaborării și coordonarea echipelor devin din ce în ce mai vitale, liderul echipei (Team Leader) ocupă o poziție centrală în asigurarea succesului și performanței acestora. Prin abilitățile și competențele sale, liderul echipei este capabil să alinieze și să dirijeze echipa spre atingerea obiectivelor organizației. Capitolul următor are ca scop investigarea rolului liderului echipei și a importanței abilităților sale, care joacă un rol esențial atât la nivelul echipei, cât și la nivelul organizației.
		\subsection{ Liderul de echipa - oportunitati si provocari}

\quad\quad\space In spatele oricarei organizatii de succes se afla persoane care isi iau un angajament fata de misiunea acesteia si care conduc alte grupuri de persoane pentru a fi siguri ca actiunile lor conduc la dezvoltarea si cresterea organizatiei. Una dintre aceste persoane este team leader-ul si inainte de a continua cu analiza activității și responsabilităților acestuia este important să adoptăm o definiție clară a acestui concept. Există mai multe definiții ale liderului de echipă, insa in cadrul acestei lucrari ne vom concentra pe definitia data de dicționarul de Engleză de Afaceri Cambridge si anume ca " team leader este persoana responsabila de o echipa".\footnote{Cambridge Business English Dictionary, meaning of team leader in English.} Prin aceasta afirmatie se intelege ca liderul de echipa poartă o responsabilitate fundamentală pentru performanța echipei și acționează ca punct central pentru luarea deciziilor, avand ca obiectiv principal atingerea obiectivelor organizatiei.

	\quad\space O parte importanta din rolul unui team leader se referă la acțiunile pe care le întreprinde în cadrul echipei pentru a asigura coordonarea și succesul acesteia. Activitatile principale implică coordonarea și gestionarea unei echipe, asigurându-se că obiectivele sunt atinse în mod eficient, in acelasi timp fiind responsabili de planificarea și organizarea activităților, delegarea sarcinilor, motivarea membrilor echipei și menținerea unei colaborări cat mai  productive. Conform studiului realizat de Frederick P. Morgeson si a colaboratorilor din anul 2010 asupra proceselor de leadership \footnote{Frederick P. Morgeson, D. Scott DeRue, Elizabeth P. Karam, Leadership in Teams: A Functional Approach to Understanding Leadership Structures and Processes,2010}, putem clasifica activitatile unui team leader in doua faze: 
	\begin{itemize}

	\item Faza de tranzitie: în care echipele se concentrează asupra activităților legate de structurarea echipei, planificarea lucrului echipei și evaluarea performanței echipei, astfel încât echipa să poată atinge în cele din urmă obiectivul sau scopul său. În timpul acestei perioade  se enumera activitati precum asigurarea unei combinații potrivite de persoane în echipă, definirea misiunii generale, a obiectivelor și a standardelor de performanță ale echipei, structurarea rolurilor și responsabilităților în echipăm, asigurarea ca toți membrii echipei să fie capabili să performeze eficient, înțelegerea mediului echipei si facilitarea proceselor de feedback în echipă. 

	\item Faza de actiune: care include porțiunea ciclului de performanță a echipei unde se concentrează pe activități care contribuie direct la atingerea obiectivelor sale. In timpul acestei faze se exercita actiuni precum monitorizarea echipei și a mediului său de performanță, gestionarea granițelor dintre echipă și mediul organizațional mai larg, provocarea echipei să se îmbunătățească în mod continuu, implicarea în efectuarea activităților echipei, rezolvarea problemelor cu care echipa se confruntă, obținerea resurselor pentru echipă, încurajarea echipei să acționeze autonom și cultivarea unui climat social pozitiv în cadrul echipei. 
	\end{itemize}

\quad\space 

%\quad\quad\space Tipuri de lideri- teorie vs practica?? 


	\quad\space Un pilon fundamental al succesului organizational care e totodata si instrumentul de baza a unui team leader este echipa. Echipa este un grup de indivizi care indeplinesc roluri specifice si interactioneaza dinamic, interdependent si adaptabil in vederea atingerii unui obiectiv comun. \footnote{Denise L. Reyes, Julie Dinh, Eduardo Salas, What Makes A Good Leader?, 2019} Echipele contemporane necesita mai multa atentie și interacțiune continuă pentru a menține o performanță ridicată pe parcursul vieții lor temporare. Team leader-ul  trebuie acum să se concentreze pe motivarea și susținerea echipelor personalizand procesul in functie de membrii echipei. Luand in considerare articolul lui Victor Sohmen,despre leadership si teamwork\footnote{ Victor Sohmen, Leadership and teamwork: Two sides of the same coin, 2013}, se pot constata opt principii care cuprind aspecte esentiale care faciliteaza legatura dintre team leader si lucrul in echipa.
	\begin{itemize}
	\item\textbf{Viziunea} este primul element pe care orice lider ar trebui sa impartaseasca de la inceput cu echipa sa. Prin implicarea membrilor echipei în dezvoltarea viziunii și asigurând o aliniere constantă cu obiectivele organizației, liderul creează un sentiment de apartenență și împuternicire în echipă.
	\item Un alt element important este \textbf{integritatea} care genereaza incredere, respect si credibilitate membrilor echipei fata de actiunile pe care le intreprinde liderul.
	\item \textbf{Comunicarea} este alt aspect important deoarece team leader-ul reprezinta legatura echipei cu organizatia. Acesta trebuie sa comunice clar obiectivele, responsabilitatile, gradul de perfomanta, nivelul de asteptari si modul de a oferi feedback.
	\item Un detaliu ce nu poate lipsi dintr-o echipa este \textbf{colaborarea}. Liderii de succes se adapteaza in functie de situatie si creaza un mediu propice in care fiecare membru al echipei are un rol bine stabilit si ofera un plus de valoare.
	\item \textbf{Orientarea către obiective} este un alt element ce are un impact mare asupra activitatilor unui lider .Pe toata perioada colaborarii team leader-ul trebuie sa tina cont de obiectivele organizatiei si sa le imbine cu obiectivele personale ale membrilor echipei. Prin implicarea comună în atingerea acestor obiective se consolidează colaborarea și succesul echipei.
	\item Un element nou este \textbf{empowermentul} care inseamna dezvoltarea increderii de sine si autonomie in membrii echipei. Acest lucru poate avea loc prin comunicare eficienta intre team leader si team members, prin delegarea responsabilitatilor la persoanele potrivite si prin sedinte de mentorat si poate contribui la construirea unor relatii pozitive si la realizarea obiectivelor echipei.
	\item Un alt aspect ce poate fi folosit de catre team leader este \textbf{creativitatea} care poate fi un avantaj daca este utilizata in mod constructiv. Prin promovarea unui mediu de lucru sanatos care incurajeaza inovatia, liderii pot imbunatati si transforma rezultatele organizatiei, astfel incat sa obtina un avantaj competitiv pe piata existenta.
	\item Ultimul element pe care s-a pus tot mai mare accentul in ultima perioada este\textbf{ team-building-ul}. Echipa incepe ca un grup de straini si trebuie sa fie sinergizata intr-o echipa de inalta performanta, iar liderul de echipa are un rol semnificativ in definirea atmosferei de la locul de munca. Echipa trebuie să fie dezvoltată într-un mod care să promoveze o cultură pozitivă de construire a echipei.
	\end{itemize}
	\quad În contextul managementului echipei, dobândirea și dezvoltarea abilităților de leadership reprezintă un aspect crucial pentru obținerea succesului insa conducerea eficientă a unei echipe nu necesită doar competențe tehnice și experiență, ci și înțelegerea profunzimii rolului de lider si aplicarea autoritatii intr-un mod cat mai echilibrat. Astfel, potrivit unui articol scris de Dr. Radhika Kapur\footnote{Dr. Radhika Kapur,Leadership Skills: Fundamental in Leading to Effective Functioning of the Organizations, 2020} , se evidențiază că un lider de echipă are nevoie de trei trăsături importante pentru a atinge succesul, și anume:

	\begin{itemize}
	\item \textbf {Dorinta de a conduce} este crucială pentru a furniza cunoștințe, motivație și soluții, promovând astfel succesul și dezvoltarea organizațională. . Fara a avea motivatia intrinseca de a conduce liderul va ajunge sa aiba o rezistenta fata de munca de conducere iar membrii echipei vor resimti acest lucru si nu va mai exista o colaborare eficienta.

	 \item \textbf{Angajamentul fata de misiunea si viziunea organizatiei} facilitează generarea unei conștientizări în ceea ce privește metodele, abordările și strategiile care vor fi puse în practică pentru a asigura funcționarea eficientă a organizației și realizarea misiunii și a obiectivelor. Misiunea și viziunea se concentrează în principal asupra a ceea ce va face organizația, cui îi va servi și cum obiectivele și funcționarea organizației vor fi benefice comunității. 

	 \item\textbf{Integritatea} este o trasatura de care liderii pot beneficia în numeroase moduri. Acestea includ îndeplinirea eficientă a responsabilităților de serviciu, formarea unor relații cordiale cu ceilalți, obținerea respectului și aprecierii, capacitatea de a lucra sub stres, abilitatea de a face față problemelor și provocărilor, precizia în îndeplinirea sarcinilor de serviciu, onestitatea și dreptatea și tratarea cu respect și politețe a celorlalți indivizi, atât interni, cât și externi organizației. Acești factori sunt benefici pentru membrii echipei și pentru organizație în ansamblu. 

\end{itemize}

\newpage
	\space\quad\quad În societatea actuală, în care evoluția și schimbarea sunt inevitabile, dezvoltarea continuă a devenit un aspect esențial pentru atingerea succesului personal și profesional. Ideea de a căuta și de a adopta o abordare orientată spre dezvoltare a devenit din ce în ce mai importantă într-o lume în care cunoștințele și competențele se depreciază rapid. La fel se intampla si in cazul liderilor de echipa care se confrunta zilnic cu noi tendinte si provocari aparute pe piata muncii. Traind intr-o era a tehnologiei, accesul la informatii nu a fost niciodata mai facil, iar sursele de imbunatatire se afla la un click distanta. În conformitate cu cercetarea efectuată de Radhika Kapur ibidem (2020)\footnote{Ibidem}, s-au evidentiat cateva obiceiuri care pot ajuta liderii de echipa sa-si imbunatateasca rolul pe care il detin. Printre ele se enumera: 
	\begin{itemize}
	\item\textbf{ Dezvoltarea pasiunii} este esentiala pentru imbunatatirea abilitatilor de conducere. Aceasta îi ajută să-și îndeplinească responsabilitățile cu pasiune și să obțină rezultatele dorite.Pe lângă beneficiile personale, dezvoltarea pasiunii în lideri are și un impact asupra subordonaților lor deoarece pot transmite entuziasm și inspirație celor din jurul lor, creând un mediu de lucru pozitiv și motivant. De asemenea, pasiunea dezvoltată în lideri îi ajută să facă față cu succes provocărilor și obstacolelor pe care le întâlnesc în activitatea lor. Fiind pasionați, aceștia sunt mai puțin predispuși la descurajare și  sa renunțe în fața dificultăților. În schimb, pasiunea îi încurajează să caute soluții creative și să găsească modalități inovatoare de a depăși obstacolele.

	\item Liderii trebuie să \textbf{recompenseze și să motiveze} angajații pentru a îmbunătăți abilitățile de leadership. Prin oferirea de recompense și stimulente, precum și prin promovarea și concediile plătite, angajații devin mai interesați și entuziasmați în îndeplinirea sarcinilor lor de serviciu. Această motivație ajută la concentrarea și angajamentul angajaților în atingerea obiectivelor organizației. Această abordare este considerată unul dintre modurile cruciale de a stimula concentrarea și angajamentul angajaților față de îndeplinirea sarcinilor de serviciu.

	\item Liderii trebuie să se angajeze în \textbf{cercetare periodică}, utilizând diverse surse precum internetul, cărți și articole, pentru a-și îmbunătăți abilitățile de leadership. Această cercetare poate include și implicarea în lucrări de teren, oferindu-le liderilor informații practice și experiențe relevante. Prin utilizarea tehnologiei și internetului, liderii pot accesa numeroase resurse și studii despre modalități de a-și îmbunătăți abilitățile de leadership.


	\item În interiorul organizațiilor, angajații se confruntă cu diverse probleme legate de sarcinile și responsabilitățile de serviciu, infrastructură, facilități, resurse și condiții de lucru. Pentru a aborda aceste probleme, este necesară \textbf{implementarea unui sistem eficient de soluționare a reclamațiilor}, care să permită angajaților să-și exprime nemulțumirile și să primească soluții adecvate. Prin asigurarea unui astfel de sistem, liderii pot promova satisfacția angajaților și pot contribui la îmbunătățirea abilităților lor de leadership.

	\item \textbf{Conștientizarea situațională} implică percepția elementelor și evenimentelor din mediu în raport cu timpul și spațiul (A Practical Guide to Situational Awareness, 2012). Liderii integrează avansurile tehnologice în sarcinile lor și folosesc metode moderne de învățare în programele de pregătire. Dezvoltarea conștientizării situaționale este esențială pentru a face față cu succes condițiilor de mediu interne și externe. 

	\end{itemize}


	\subsection{Abilitati si competente specifice team leader-ului}
	
	\quad\quad În fiecare organizație, exista diverse tipuri de lideri.In literatura de specialitate, exista doua tipuri principale de lideri si anume: informali, exercitând o influență semnificativă asupra colegilor datorită personalității și calităților sale si formali,  ocupând o poziție oficială cu putere și autoritate de a lua decizii , însă ambele tipuri împărtășesc abilități care îi ajută să atingă obiectivele personale sau ale organizației. Aceste abilități de leadership pot fi considerate instrumente, comportamente și capacități esențiale pentru a promova bunăstarea angajaților și a îmbunătăți performanța organizației.\footnote{MTD Training, Leadership Skills, 2010} Indiferent de poziția lor, liderii își focalizează eforturile în direcționarea și motivarea angajaților pentru a atinge sarcinile și obiectivele stabilite. Prin dezvoltarea potențialului individual al celorlalți, liderii autentici demonstrează o abilitate de leadership remarcabilă. Pentru a avea succes în implementarea acestor abilități, liderii trebuie să faciliteze dezvoltarea competențelor celorlalți, adaptându-se la diverse situații și asigurând beneficii atât pentru angajați, cât și pentru organizație.

	\subsubsection{Diferente intre abilitatile unui manager si unui lider}

	
	







\newpage

	\section*{Referințe bibliografice}
	\addcontentsline{toc}{section}{\textsc{Referințe bibliografice}}
	\space
	\bigskip
	\bigskip

	\textbf{Cărți în format electronic:}
	

	\textbf{Articole științifice în format electronic:}
	\begin{enumerate}[1.]
		\item Frederick P. Morgeson, D. Scott DeRue, Elizabeth P. Karam, Leadership in Teams: A Functional Approach to Understanding Leadership Structures and Processes,2010
	\end{enumerate}

	\textbf{Articole preluate de pe pagini web:}












\end{document}